\documentclass[aspectratio=169]{beamer}
\usepackage[utf8]{inputenc}

\usepackage{utopia} %font utopia imported

\usetheme{Madrid}
\usecolortheme{default}

%------------------------------------------------------------
%This block of code defines the information to appear in the
%Title page
\title[Minesweeper Solver] %optional
{Minesweeper Solver}

\subtitle{Project proposal}

\author[Team 4] % (optional)
{Chunxu Guo \and Jiahao Huang \and Qianyu Liu\\ \and Chenyang Zhang \and Tianyi Zhang}

\institute[ShanghaiTech] % (optional)
{
  ShanghaiTech University
}

\date[Dec. 14] % (optional)
{CS181: Artificial Intelligence I, Fall 2020}

\logo{\includegraphics[height=1.0cm]{figs/icon.png}}

%End of title page configuration block
%------------------------------------------------------------



%------------------------------------------------------------
%The next block of commands puts the table of contents at the 
%beginning of each section and highlights the current section:

\AtBeginSection[]
{
  \begin{frame}
    \frametitle{Table of Contents}
    \tableofcontents[currentsection]
  \end{frame}
}
%------------------------------------------------------------


\begin{document}

%The next statement creates the title page.
\frame{\titlepage}


%---------------------------------------------------------
%This block of code is for the table of contents after
%the title page
\begin{frame}
\frametitle{Table of Contents}
\tableofcontents
\end{frame}

%---------------------------------------------------------

\section{Topic and Motivation}
% Tianyi Zhang





%---------------------------------------------------------

\section{Logic Inference}
% Qianyu Liu





%---------------------------------------------------------

\section{SAT Solver}
% Chunxu Guo





%---------------------------------------------------------

\section{CSP Probability Model}
% Jiahao Huang

\begin{frame}
	\frametitle{Minesweeper CSP Model}
	With current observation $K$ (i.e. the knowledge matrix), we can enumerate all configurations compatible with $K$, then
	\begin{itemize}
		\item deduce for each cell $c$ whether it is safe (no configuration where $c$ is mined)
		\item deduce for each cell $c$ whether it is mined ($c$ is mined in all configurations)
		\item compute the probability for $c$ to be mined as the ratio $$ p_m(c) = \frac{\mbox{\# (cfgs where c is mined)}}{\mbox{\# (all cfgs)}} $$

	\end{itemize}
\end{frame}

\begin{frame}
	\frametitle{CSP Formulation}
	Searching for a single configuration compatible with $K$ can be formulated as a CSP as follows:

	\begin{itemize}
		\item ${\forall}c=(i,j),\ x_{i,j} \in \{0,1\}$ \hfill ($x_{i,j}$ is either mined or safe)
		\item ${\forall}c=(i,j),\ (K_{i,j} \geq 0) \Rightarrow (x_{i,j}=0)$ \hfill (a cell $c$ with non-negative value is safe)
		\item ${\forall}c=(i,j),\ (K_{i,j} \geq 0) \Rightarrow (\sum_{c^{\prime}=(i^{\prime},j^{\prime}) \in a(c)} x_{i^{\prime},j^{\prime}} = K_{i,j})$ \\ \hfill (the non-negative value on a cell c is the number of adjacent mined cells)
		\item $\sum_{c=(i,j)} x_{i,j} = m$ \hfill (there are exactly m mined cells)
	\end{itemize}
\end{frame}

\begin{frame}
	\frametitle{CSP Optimization}
	Inference problem in Minesweeper is a hard problem as it has been proven to be NP-complete [Kaye, 2000]. This justifies developing efficient resolution techniques.

	\begin{itemize}
		\item Some simple rules, as are already mentioned in the Logic Inference part, allow determining mined or safe cells quickly, and CSP Inference can be used after that.
		\item Of all the uncovered cells, the interior part does not provide any information on where mines may be lying. In other words, only the boundary should be take into consideration.
		\item The values observed on the boundary are only related to mines in the fringe, so we can restrict the reasoning about constraints induced to the fringe.
		\item More to be discovered ...
	\end{itemize}
\end{frame}

%---------------------------------------------------------

\section{POMDP View}
% Chenyang Zhang

\begin{frame}
	\frametitle{POMDP Model}
	POMDP: Partially Observable Markov Decision Process
	\begin{itemize}
	    \item Generalization of a Markov decision process (MDP)
	    \item Agent cannot directly observe the underlying state
	    \item Maintain a probability distribution over the set of possible states
	\end{itemize}
\end{frame}

\begin{frame}
	\frametitle{Minesweeper POMDP Model}
	Minesweeper game can be modeled as a POMDP $<S, S_e, A, T, R, O, \Omega, b_0>$ where:

	\begin{itemize}
		\item set of states $S$: init state, normal states, failure state
		\item terminal state $S_e$: success state, failure state
		\item actions in $A$: try hidden cell $c$
		\item transition function $T$
		\item reward $R(s, a, s')$
		\item observations in $O$
		\item observation function $\Omega$: updates the knowledge matrix according to the last action
		\item $b_0$: initial probability distribution over states
	\end{itemize}
\end{frame}

\begin{frame}
	\frametitle{POMDP Challenges}
	Belief space is huge: 
	\begin{itemize}
		\item $2^{W \times H}$ states!
		\item Solving POMDPs exactly is computationally intractable
		\item MOMDP: Mixed Observability Markov Decision Process
		\begin{itemize}
			\item we can derive a compact lower-dimensional representation of the belief space
		\end{itemize}
		\item Monte-Carlo Tree Search
	\end{itemize}
\end{frame}


%---------------------------------------------------------

\section{CNN Solver}
% Tianyi Zhang





%---------------------------------------------------------

\section{\LaTeX ~ example section}

%---------------------------------------------------------
%Changing visivility of the text
\begin{frame}
\frametitle{Sample frame title}
This is a text in second frame. For the sake of showing an example.

\begin{itemize}
    \item Text 1
    \item Text 2
    \item Text 3
    \item Text 4
\end{itemize}
\end{frame}

%---------------------------------------------------------

%Example of the \pause command
\begin{frame}
In this slide \pause

the text will be partially visible \pause

And finally everything will be there
\end{frame}

%---------------------------------------------------------

%Highlighting text
\begin{frame}
\frametitle{Sample frame title}

In this slide, some important text will be
\alert{highlighted} because it's important.
Please, don't abuse it.

\begin{block}{Remark}
Sample text
\end{block}

\begin{alertblock}{Important theorem}
Sample text in red box
\end{alertblock}

\begin{examples}
Sample text in green box. The title of the block is ``Examples".
\end{examples}
\end{frame}
%---------------------------------------------------------


%---------------------------------------------------------
%Two columns
\begin{frame}
\frametitle{Two-column slide}

\begin{columns}

\column{0.5\textwidth}
This is a text in first column.
$$E=mc^2$$
\begin{itemize}
\item First item
\item Second item
\end{itemize}

\column{0.5\textwidth}
This text will be in the second column
and on a second tought this is a nice looking
layout in some cases.
\end{columns}
\end{frame}
%---------------------------------------------------------


\end{document}
